\subsection*{Entendiendo: ¿Qué es un \textit{stock} en tarjetas de crédito?}

El objetivo es predecir cuántas tarjetas físicas (aún no activadas) deben fabricarse en cada periodo. Para esto, debemos tener en cuenta tres factores fundamentales:

\begin{enumerate}
	\item \textbf{Tarjetas físicas sin activar:} Son los objetos físicos que se deben tener listos para satisfacer la demanda futura. Su producción debe anticiparse al uso, pero sin caer en el exceso.
	
	\item \textbf{Demanda proyectada:} Depende de múltiples variables, como campañas de marketing (e.g. tarjetas temáticas), comportamiento regional y perfil de los usuarios (por edad, ingresos, historial).
	
	\item \textbf{Costos de sobreproducción o desabasto:} Sirven como función objetivo en la optimización. También se deben considerar costos de almacenamiento, distribución y tiempos de entrega.
\end{enumerate}

Esto nos lleva naturalmente a un problema de optimización bajo incertidumbre, con un componente predictivo fuerte (forecast) y decisiones estructurales que se prestan al modelado matemático.
